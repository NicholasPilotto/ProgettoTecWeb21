\documentclass{article}

\usepackage{graphicx}
\usepackage[utf8]{inputenc}
\usepackage[italian]{babel}
\usepackage{fancyhdr}
\usepackage{hyperref} % Pacchetto per gli url e link nel documento
\usepackage{longtable}
\usepackage[table]{xcolor}
\usepackage{makecell}
\usepackage{geometry}
\usepackage{array}
\usepackage{lastpage}
\usepackage{float}


% Definisce un nuovo tipo di colonna per le tabelle. E' come p{}, ma con il testo all'interno della cella già centrato.
\newcolumntype{P}[1]{>{\centering\arraybackslash}p{#1}}

% Setup pacchetto "hyperref"
\hypersetup{
	colorlinks=true,
	linkcolor=black,
	urlcolor=blue
}

\setlength{\headheight}{26.87454pt}
\addtolength{\topmargin}{-14.87454pt}

\usepackage{enumitem}

% Definisce i margini e il formato delle pagine
\geometry{a4paper, top=3cm, left=2.5cm, bottom=3cm, right=2.5cm}

% AGGIUNTA PARAGRAFI E SOTTOPARAGRAFI
\setcounter{secnumdepth}{5}
\setcounter{tocdepth}{5}

% INDENTAZIONE SOTTOPARAGRAFI
\makeatletter
\renewcommand\subparagraph{
\@startsection{subparagraph}{5}{0mm}{-\baselineskip}{.5\baselineskip}{\normalfont \normalsize \bfseries }}
\makeatother

% INDENTAZIONE PARAGRAFI
\makeatletter
\renewcommand\paragraph{
\@startsection{paragraph}{4}{0mm}{-\baselineskip}{.5\baselineskip}{\normalfont \normalsize \bfseries }}
\makeatother

\title{SecondRead}
\date{2021/2022}

\begin{document}
\pagenumbering{gobble}

\makeatletter
    \begin{titlepage}
        \begin{center}
            \includegraphics[width=0.4\linewidth]{images/icon.jpg}\\[10ex]
            {\huge \bfseries  \@title }\\[5ex]
            {\large \@date} \\[10ex]
            {\large \textit{Componente:} Nicholas Pilotto - 1230237} \\[2ex]
            {\large \textit{Componente:} Michele Bonavigo - } \\[2ex]
            {\large \textit{Componente:} Mattia Casarotto - } \\[2ex]
            {\large \textit{Componente:} Anita Meta - } \\[2ex]
            {\large \textit{Destinatari:} Prof.ssa\ Ombretta Gaggi} \\[2ex]
        \end{center}
    \end{titlepage}
\makeatother
\thispagestyle{empty}
\newpage

% Head pagine dopo la prima (con stile fancy)
\pagestyle{fancy}
\fancyhf{}
\rhead{\textbf{SecondRead}}
\lhead{\includegraphics[width=1.1cm]{images/icon.jpg}}

\setcounter{table}{0} % Resetta il contatore delle tabelle

\tableofcontents \pagebreak

% Foot pagine dopo gli indici
\pagenumbering{arabic}
\rfoot{Pagina \thepage/\pageref*{LastPage}}

\newpage
\section{Abstract}\label{sec:abstract}

\subsection{Il progetto}
L’idea dietro \textit{Second Read}, svolto come progetto del corso \textit{Tecnologie Web}, nell’anno accademico 2021-2022, è stata quella di creare un’ampia piattaforma per la vendita di libri di vario genere.

Gli appassionati possono trovare ed acquistare qui tutti i libri di loro interesse, oppure semplicemente esplorare le diverse opzioni offerte.

Il nome \textit{“SecondRead”} è stato scelto per dare l’idea di fondo di questo progetto: un spazio \textit{web} dove ognuno possa trovare il libro che tanto vuole leggere (\textit{Read}) e dove tornerà di sicuro a visitare per una seconda volta (\textit{Second}).

La pagina è stata pensata e progettata in modo interattivo, soddisfando qualsiasi bisogno dell'utente. All'interno, i libri sono stati classificati in diverse categorie in base al loro contenuto. Un utente o anche un visitatore casuale non registrato può cercare un libro nella pagina Ricerca tramite il suo titolo, nome dell'autore oppure utilizzando direttamente il suo codice \textit{ISBN}.

A tutti i visitatori viene offerto la possibilità di registrarsi gratuitamente e creare il proprio \textit{account} personale, con il quale possono poi interagire col sito e sfruttare le diverse funzionalità offerte.

In modo da rendere \textit{SecondRead} un sito \textit{web} vero e proprio, durante la fase di sviluppo si è voluto prestare grande attenzione alle componenti principali di una pagina come l'utilizzabilità, il rispetto degli standard \textit{W3C} e la separazione adeguata tra struttura, presentazione e comportamento tenendo sempre conto delle regole di accessibilità.
\newpage
\section{Analisi}

\subsection{Analisi delle caratteristiche utenti}
\textit{SecondRead} rappresenta una piattaforma intermediaria per la vendita dei libri tra individui e le grandi case editrici. L’utenza primaria del sito sarà quindi composta da lettori, siano essi occasionali o appassionati.

Questo non esclude tuttavia il fatto che anche un soggetto legato all’ambito accademico come un professore oppure uno studente possa utilizzare la pagina per trovare libri riguardanti il proprio corso di studio.

Cercando di includere ogni ambito di interesse per i lettori, sono state create inizialmente circa 12 diverse categorie di libri. Si cerca sempre di avere una quantità sufficiente di copie disponibili per ogni libro in modo da poter offrire agli utenti del sito un servizio impeccabile.

Il sito offre all’utente la possibilità di svolgere una ricerca veloce e precisa, effettuata di solito dalle persone che sanno già cosa vogliono, ma anche la possibilità di svolgere ricerche ad ampio spettro che riguardano argomenti correlati a quello inizialmente cercato.

Un altro obiettivo è stato quello di includere nel \textit{target group} tutte le diverse fasce d’età, motivo per il quale si è optato per un linguaggio semplice, informale e intuitivo che possa essere comprensibile a tutti.

La stessa filosofia è stata applicata anche per quanto riguarda struttura e layout del sito: esso è veloce, intuitivo e responsive in modo da adattarsi a vari tipi di dispositivi. Tutto questo rispettando le principali convenzioni del web.

\subsection{Possibile ricerche sui motori di ricerca}
Elencate di seguito, in ordine di rilevanza (da particolare al generale), le possibili ricerche che dovrebbero presentare tra i risultati il sito web.

La ricerca a cui deve sicuramente rispondere il sito è quella contenente il nome stesso, ossia \textit{SecondRead}. Queste ricerche saranno maggiormente effettuate da utenti del sito, da persone che lo hanno visitato precedentemente oppure da persone a cui è stato raccomandato o riferito, ad esempio da un amico.

Essendo un sito di libri online, è fondamentale che risponda in una ricerca a tutte le \textit{query} che contengono un \textit{ISBN, un titolo di libro, oppure un nome di autore o casa editrice} dei libri che sono presenti nella pagina. Questo tipo di ricerca viene effettuata dalle persone che stanno cercando un particolare libro come per esempio uno studente, un genitore o anche un professore universitario.

È importante che il sito risponda anche alle ricerche ad ampio spettro. Ad esempio, se un appassionato di scienza cerca “Scienza e Fantascienza” oppure un appassionato dei manga cerca \textit{“Fumetti e Manga”}, essendo queste categorie di libri parte del sito, esso dovrebbe apparire tra i risultati della ricerca.

Uno degli obiettivi principali del team è quello di espandere la clientela ed attirare nuovi utenti. In modo da realizzarlo, è necessario che il sito risponda anche alle ricerche generali che possono contenere parole chiave come: \textit{libri, libro online, vendita libri, dove comprare libri online,..} . Queste sono le ricerche che vengono effettuate di solito da persone che non hanno esperienza e preferenze specifiche ma che vogliono iniziare a leggere oppure hanno recentemente sviluppato la lettura come \textit{hobby}.

Un altro caso che aiuterebbe ad aumentare la visibilità, è quello in cui la ricerca contiene parole chiave come: libri bestseller, libri scontati online, nuove uscite, prezzo sotto 5 euro...

Una persona che vorrebbe fare un regalo senza spendere troppo sarebbe il caso di utilizzo di questo tipo di ricerca.

In generale, essendo il sito progettato in modo da poter essere utilizzato da tutte le fasce d'età, si è prestato attenzione alle ricerche effettuabili da utenti appassionati ed esperti, ma anche a quelle effettuate da principianti.

\subsection{Conclusioni}
Partendo dall’analisi e dalle considerazioni fatte precedentemente riguardanti il sito e gli utenti, otteniamo le seguenti conclusioni:
\begin{itemize}
	\item \textit{Utenza finale}: gli appassionati di libri e persone legate all'ambito 	accademico
	\item \textit{Target group}: tutte le varie fasce d’età
	\item Ricerca: è possibile effettuare vari tipi di ricerca. Nel caso di ricerca
			specifica, si può utilizzare direttamente la barra di ricerca, mentre per
			quella ad ampio spettro si possono trovare informazioni nei raggruppamenti di
			generi o fasce di prezzo
	\item \textit{Interazione}: all’interno del sito è possibile interagire in diversi
			modi. Un esempio di
			inserimento di dati si trova nella creazione di un account oppure 
			nell'aggiunta di un indirizzo. Un esempio di cancellazione di dati lo 
			proviamo nella cancellazione di recensioni memorizzate precedentemente.
			Si ha anche la possibilità di modificare i dati inseriti come la password,
			l’username oppure l’email.
	\item \textit{Community}: le recensioni sono un elemento pensato ed aggiunto al sito 	in modo da poter creare una forma solida di comunicazione ed interazione tra 	gli utenti. Leggendo le diverse opinioni presenti sulla pagina di un
			determinato libro, l’utente crea un'idea generale più chiara del libro e
			decide se acquistarlo.
\end{itemize}

\subsection{Utenti}
Abbiamo 3 diverse tipologie di utenti:
\begin{enumerate}
	\item \textit{Utente generico(generale)}: Un utente viene considerato generico quando:
		\begin{itemize}
			\item Non è stato registrato, ossia non ha creato il suo account sul sito.
			\item Possiede un account ma non ha ancora effettuato l’accesso.
		\end{itemize}
		Questa tipologia di utente ha accesso a buona parte del sito ma subisce restrizioni su alcuni servizi offerti. Un utente generico non può, ad esempio, fare ordini oppure lasciare recensioni. Tuttavia è comunque in grado di:
			\begin{enumerate}[label*=\arabic*.]
				\item Visualizzare la Homepage del sito
				\item Visualizzare la pagina “Ricerca”
				\item Visualizzare la pagina “Generi”
				\item Visualizzare la pagina “Bestseller”
				\item Visualizzare la pagina “Offerte”
				\item Visualizzare l’elenco dei libri listati in ogni categoria
				\item Cercare un libro nella pagina di ricerca
				\item Visualizzare tutti i dettagli di un particolare libro in vendita 
				\item Registrarsi
				\item Effettuare l’accesso
			\end{enumerate}
	\item \textit{Utente loggato}: Rappresenta l’utente che ha effettuato l’accesso al suo account personale. Eredita tutti i casi d’uso dell’utente generico tranne la parte di registrazione e dell’accesso. Dispone di altre funzionalità aggiuntive come:
			\begin{enumerate}[label*=\arabic*.]
				\item Visitare l'Area Privata 
				\item Visualizzare i suoi ordini
				\item Gestire gli indirizzi
				\item Gestire le recensioni
				\item Visualizzare e modificare le informazioni di login 
				\item Gestire ed aggiornare la propria \textit{wishlist}
				\item Effettuare un ordine
				\item Lasciare una recensione
				\item Contattare il numero di assistenza in caso di problemi
			\end{enumerate}
	\item \textit{Amministratore(admin)}: Rappresenta l’utente che gestisce il sito ed è identificato dall’indirizzo email: \underline{admin@gmail.com}. Possiede la maggior parte delle funzionalità degli utenti sopra elencati ma con alcune eccezioni. Non può ad esempio:
		\begin{itemize}
			\item Lasciare una recensione
			\item Aggiungere elementi nella \textit{wishlist} oppure nel carrello
			\item Effettuare un ordine
		\end{itemize}
Dopo aver effettuato l’accesso al suo account, attraverso l’Area Privata sulla navbar, può eseguire le seguenti funzionalità:
		\begin{enumerate}[label*=\arabic*.]
			\item Aggiungere nuovi libri
			\item Modificare dettagli di un libro già presente nella pagina 
			\item Visualizzare i dati analitici del sito
		\end{enumerate}
\end{enumerate}
\newpage
\section{Struttura del sito}\label{sec:struttura}

\subsection{Header}
Il \textit{header} è sempre lo stesso ed è presente su tutte le pagine. Contiene il logo e il nome del sito in alto a sinistra.

\subsection{Breadcrumb}
La \textit{breadcrumb} è uno strumento fondamentale per quanto riguarda l’accessibilità. La funzione principale è quella di aiutare l'utente ad orientarsi sulla propria posizione all'interno della pagina \textit{web}. Si trova immediatamente sotto il menu ed è presente su tutte le pagine, sia nella versione \textit{desktop} sia in quella \textit{mobile}.

Contiene i \textit{link} relativi alle pagine che sono state visitate precedentemente dall’utente. L'ultima pagina listata è quella corrente che non contiene un \textit{link} in modo da evitare una circolazione non necessaria.

\subsection{NavBar}
La \textit{navbar} costituisce una delle parti più importanti del sito siccome contiene i vari link per la navigazione. Nella maggior parte delle pagine, \textit{header} e \textit{navbar} sono invariati ma la cosa cambia in base al tipo di utente.
	\begin{enumerate}
		\item Utente generico: nel caso in cui l’utente non possiede un \textit{account} oppure non ha ancora effettuato l’accesso, la \textit{navbar} è costituita dalle pagine:
			\begin{itemize}
				\item Ricerca
				\item Generi
				\item Bestseller
				\item Offerte
				\item Area Riservata
			\end{itemize}
		Area Riservata non è in sé una pagina come le altre, ma serve a creare il collegamento con la pagina Accedi. L’utente generico, dopo aver cliccato su Area Riservata, viene indirizzato alla pagina Accedi dalla quale può direttamente effettuare l'accesso all'\textit{account}.

		In alternativa, se l’utente generico non possiede un’account \textit{SecondRead} (non lo ha creato precedentemente), può farlo direttamente cliccando il pulsante Registrati presente nella pagina Accedi. Verrà indirizzato alla pagina Registrati dove deve compilare il \textit{form} per la registrazione.

		Si è scelto di aggiungere un ulteriore livello di profondità (per andare alla pagina Registrati si deve effettuare un “\textit{click}” in più rispetto a quello che si doveva fare se la pagina fosse stata direttamente presente sulla \textit{navbar}) con lo scopo di evitare il sovraccaricamento del \textit{menu} e soprattutto in modo da evitare il sovraccarico cognitivo dell’utente.

		Allo stesso tempo, attraverso questa organizzazione l'adattamento della \textit{navbar} per i dispositivi con schermi più piccoli risulta più facile.

		\item \textit{Utente loggato}: l’utente ha già effettuato l’accesso sul proprio account. La sua \textit{navbar} è composta dalle pagine:
			\begin{itemize}
				\item Ricerca
				\item Generi
				\item Bestseller
				\item Offerte
				\item Area Riservata
				\item Carrello
				\item Esci
			\end{itemize}
		In questo caso, l’Area Riservata esegue una funzione diversa. Quando l’utente lo clicca, viene indirizzato alla pagina \textit{Account} dove si trovano tutte le sezioni e funzionalità che possiede.
		\item \textit{Amministratore}: in questo caso la \textit{navbar} è leggermente diversa da quella dell'utente loggato siccome non contiene la pagina Carrello. La funzione dell'Area Riservata rimane la stessa, ossia dopo il \textit{login} se cliccata indirizza alla pagina \textit{Account} dell'amministratore (admin).
	\end{enumerate}
	
\subsection{Contenuto della pagina}
Le informazioni più importanti sono direttamente presenti nella home page siccome è la parte del sito che l’utente vede appena arrivato. Abbiamo le sezioni Bestseller, Nuove uscite e A meno di 5 euro.
Si è utilizzato lo schema a tre pannelli che aiuta a dare risposta alle 3 domande importanti:
\begin{enumerate}
	\item \textit{Di cosa si tratta?}: La risposta a questa domanda si trova nel contenuto della pagina home
	\item \textit{Dove sono?}: La prima risposta a questa domanda si trova nel titolo della pagina ma si ottiene
		una risposta anche guardando nel \textit{breadcrumb}
	\item \textit{Dove posso andare?}: La risposta si trova sul menu dove sono elencate tutte le altre pagine che
		è possibile visitare.
\end{enumerate}	
	
	
	
	
	
	
	
	
	
	
	
	
	
	
	
\newpage
\section{Progettazione del sito}\label{sec:progettazionesito}

\subsection{Obiettivi}
La fase di progettazione e sviluppo del sito \textit{SecondRead} è stata fatta tenendo sempre in conto alcuni obiettivi e principi essenziali di un sito ben organizzato e accessibile.
\begin{itemize}
	\item \textit{Separazione contenuto, presentazione e struttura}: Uno degli obiettivi più importanti nella progettazione. 
	
		Il contenuto è stato realizzato tramite documenti \textit{HTML5} e \textit{PHP}. La parte della presentazione grafica è realizzata attraverso i file esterni \textit{CSS} che vengono poi richiamati nelle pagine \textit{HTML}. La realizzazione del comportamento avviene tramite gli \textit{script} \textit{Javascript}. Tutto il codice delle pagine è stato scritto secondo le raccomandazioni \textit{W3C} e in ogni fase della progettazione si è verificato che tutto andasse bene tramite la validazione.
	\item \textit{Accessibilità}: L’accessibilità è uno dei punti fondamentali da tenere in mente durante la progettazione. Si deve realizzare un sito che possa essere utilizzabile da tutte le categorie di utenti e da tutti i tipi di dispositivi e motori di ricerca. In modo da raggiungere questo obiettivo, abbiamo utilizzato varie tecniche.
		\begin{enumerate}
			\item Sono state fornite equivalenti testuali leggibili dallo \textit{screen reader} per tutti i tipi di media come supporto per gli utenti con disabilità visiva.
			\item Grazie all’utilizzo delle misure relative o in percentuale nei fogli \textit{CSS} otteniamo un sito con un design fluido dove ogni pagina è responsive, ossia adattabile ai schermi di diverse dimensioni utilizzati dagli utenti.
			\item Sono stati usati gli attributi \textit{lang} per le parole in lingue diverse dall’italiano in modo da essere lette correttamente dallo \textit{screen reader}.
			\item La scelta dei colori del sito è stata fatta in modo da garantire un contrasto adeguato tra i diversi elementi strutturali e facilitare la lettura del contenuto a chi soffre di disturbi visivi.
			\item Con lo scopo di rendere il sito facile da usare, e cercando di aiutare l’utente alla creazione di una mappa mentale, si sono rispettate alcune convenzioni importanti del web come il colore dei \textit{link}. I \textit{link} non visitati sono sottolineati e in color blu mentre quelli visitati diventano viola.
			\item Si è aggiunto il testo \textit{“Ti trovi in :”} che precede in \textit{breadcrumb} per aiutare gli utenti svantaggiati.
			\item Abbiamo aggiunto un pulsante in basso chiamato \textit{“Torna su”} che se cliccato, effettua uno \textit{scroll} verso l’alto fino all’inizio della pagina, migliorando così l’esperienza durante lo \textit{scroll} delle pagine lunghe.
			\item Utilizzo del \textit{tag abbr} per definire le abbreviazioni.
			\item Si è utilizzato un linguaggio semplice, informale e intuitivo da poter essere
				comprensibile a tutti e in modo da aiutare l’utente a navigare all’interno della pagina corrente oppure tra le diverse pagine.
		\end{enumerate}
	\item \textit{Layout}: Il sito è stato progettato con un layout a tre pannelli:
		\textit{Header}, \textit{Navbar}, Contenuto.
	\item \textit{Resize Mobile}
	\item \textit{Schema organizzativo}
\end{itemize}
\newpage
\section{Implementazione}\label{sec:implementazione}

\subsection{Linguaggi e strumenti}
\subsubsection{HTML}
La struttura e il contenuto del sito sono stati realizzati attraverso il linguaggio \textit{HTML5}. Questa scelta è stata fatta per due principali ragioni:
\begin{enumerate}
	\item Trattandosi di un sito di vendita online, è importante che tutte le pagine siano le più coerenti possibile in modo da poter competere nel mercato con un sito fatto bene ed aggiornato. Inoltre, avere pagine scritte con linguaggi recenti aiuta moltissimo nella leggibilità del codice ed è più facile modificarlo.
	\item Un altro motivo per il quale si è scelto \textit{HTML5} è relativa all'utenza finale. Essendo principalmente le generazioni più giovani quelle ad affidarsi al mondo online per fare acquisti, i dispositivi e i motori di ricerca utilizzati saranno tendenzialmente più recenti.
	\end{enumerate}
	Si è comunque mantenuta la compatibilità con \textit{XHTML}, permettendo al sito di essere pienamente funzionante anche sui \textit{browser} più vecchi ed obsoleti. Durante tutto il processo di progettazione e creazione delle pagine si sono utilizzate le linee guida del corso di Tecnologie Web e quelle del \textit{W3C}. Il codice è stato validato utilizzando il tool di validazione del \href{https://validator.w3.org/}{W3C}.
Alcune delle regole più importanti sono:
\begin{itemize}
	\item \textbf{Separazione della struttura, presentazione e contenuto}: Non ci devono essere file \textit{script} oppure fogli di stile nel codice \textit{HTML}. Questi devono essere scritti in file esterni e poi importati nell’\textit{header} delle pagine \textit{HTML}.
	\item \textbf{Tag}: Tutti i \textit{tag} devono essere chiusi e si devono utilizzare i \textit{tag} che migliorano l'accessibilità del sito.
	\item \textbf{Metatag}: L’inserimento dei \textit{metatag} nella sezione head è molto importante in quanto migliora l’accessibilità del sito verso i browser. Allo stesso tempo, l’utilizzo corretto delle \textit{keyword} aiuta in un ranking migliore nella ricerca.
	\item \textbf{Tabelle}: Le tabelle devono essere sempre evitate, oppure rese accessibili se il loro utilizzo è essenziale.
\end{itemize}

\subsection{PHP}
Il lato server del sito viene gestito da file \textit{PHP} che stabiliscono il comportamento generale delle pagine, interagiscono con il \textit{database} e creano le sessioni di utilizzo per gli utenti loggati. Per agevolare l’interazione con il \textit{database}, sono state create delle classi di supporto che vanno a modellare i record delle principali tabelle del \textit{database}.

\textbf{File dp.php}: Questo file contiene le classi e le funzioni che fanno da intermediari tra gli script PHP e le chiamate al database. Si occupa principalmente di:
	\begin{itemize}
		\item Gestire la connessione con il \textit{database}
		\item Fornire funzioni per il recupero dei dati dal \textit{database}
		\item Fornire funzioni per l’inserimento, la modifica e la cancellazione di dati nelle tabelle
	\end{itemize}
	
\textbf{File response\_manager.php}: Contiene la classe \textit{response\_manager} che fornisce metodi necessari per la gestione delle risposte e il controllo degli errori.

\textbf{"nomepagina".php:} (esempio: \textit{bestseller.php, generi.php}) sono i file che costruiscono le pagine del sito. Utilizzano le rispettive pagine html (es: \textit{bestseller\_php.html}) che contengono l’effettivo \textit{html}.

Quasi tutte hanno una procedura di esecuzione standard:
\begin{enumerate}
	\item Verificano i permessi dell’utente
	\item Verificano la disponibilità di connessione al \textit{database}
	\item Richiedono le informazioni necessarie per la operazione da effettuare al \textit{database}
	\item Eseguono l'operazione della modifica del database richiesta dall’utente che può essere un inserimento, modifica o cancellazione di dati
\end{enumerate}

In alcune pagine vengono effettuate delle validazioni sintattiche e di dominio prima di eseguire l’inserimento o la modifica dei dati. La validazione sintattica cerca di individuare la presenza di errori che potrebbero impattare sull'esito della \textit{query}, mentre quelle di dominio mirano ad evitare l’inserimento di informazioni non coerenti nel database (esempio: aggiungere più articoli al carrello della quantità disponibile per ciascuno).

Un altro aspetto importante sul quale ci siamo focalizzati è la sicurezza dell'area privata. Sono stati aggiunti controlli che non permettono a tipi di utenti diversi di accedere ai contenuti di un altro tipo. Quindi un utente loggato non può accedere ai contenuti dell’admin e l’admin non può accedere ai contenuti di un utente registrato.

Attraverso questo controllo si riesce inoltre ad impedire l’accesso all’area riservata senza aver effettuato il \textit{login}.

\subsection{CSS}
Il \textit{layout} del sito è stato modellato utilizzando la versione 3 del linguaggio di formattazione \textit{CSS}. Come descritto anche nel caso del \textit{HTML5}, essendo gli utenti principali del sito persone relativamente giovani, possiamo assumere l’utilizzo dei motori di ricerca moderni nella maggioranza dei casi.

Sono stati prodotti 4 fogli di stile differenti, ciascuno per un opportuno dispositivo: \textit{style.css, medium.css , mini.css e print.css} (per la stampa) che contengono all'interno tutte le clausole relative alla formattazione dei documenti e varie regole disponibili con \textit{CSS3} come: \textit{flexbox}, variabili, selettori ecc.

Come strategia di progettazione abbiamo usato \textit{Responsive Web design} basata sul concetto di \textit{media query} che ottimizzano la visualizzazione in base a dispositivi specifici e dimensioni di \textit{viewport}. Grazie all’utilizzo di misure sempre \textit{relative} o in \textit{percentuale} è stato implementato un design fluido e scalabile, che permette una corretta visualizzazione delle pagine su tutti i formati di schermo, senza intaccare in alcun modo la navigazione tramite \textit{screen reader}. Così facendo si migliora anche l’accessibilità del sito.

\subsection{SQL}
Il linguaggio SQL è stato utilizzato per la codifica del \textit{database} il quale è composto dalle seguenti tabelle:
\begin{itemize}
	\item \textbf{libro:} la tabella contenente tutti i libri che sono stati aggiunti dal admin e che sono ancora in vendita. I campi sono: \textit{isbn, titolo, editore, pagine, prezzo, quantità, data\_pubblicazione, percorso, trama} che descrive brevemente di cosa parla il libro
	\item \textbf{editore:} contiene i \textit{nomi} delle diverse case editrici e l’\textit{id}. Ci possono essere più di un libro pubblicati dalla stessa casa
	\item \textbf{autore:} contiene l’\textit{id, nome e cognome} dell'autore del libro. Un libro può avere più di un autore e un autore può aver pubblicato più di un libro
	\item \textbf{pubblicazione:} è la relazione tra un libro e il suo/suoi autori. Contiene i campi \textit{libro\_isbn} e \textit{autore\_id}.
	\item \textbf{categoria:} contiene la lista delle categorie (generi) presenti. Campo \textit{id\_categoria} e \textit{nome}
	\item \textbf{appartenenza:} è la relazione tra un libro e la categoria in cui fa parte. Ci sono libri che fanno parte in più di una categoria. Campi \textit{libro\_isbn} e \textit{codice\_categoria}
	\item \textbf{offerte:} contiene la lista dei libri che sono in offerta. Campi \textit{libro\_isbn, data\_inizio, data\_fine}, e \textit{sconto} che indica la percentuale dello sconto applicato
	\item \textbf{utente:} contiene i campi in cui si salvano i dati degli utenti registrati e dell'amministratore; \textit{codice\_identificativo, nome, cognome, data\_nascita, username, email, password, telefono}
	\item \textbf{ordine:} contiene \textit{codice\_univoco, cliente\_codice} che è il \textit{codice\_identificativo} dell’utente che ha effettuato l’\textit{ordine, data, data\_partenza, data\_consegna, indirizzo, totale}
	\item \textbf{composizione:} è la relazione tra la tabella ordini e libro. Contiene i campi \textit{elemento, codice\_ordine, quantità}
	\item \textbf{wishlist:} dove vengono memorizzati gli elementi aggiunti alla wishlist, contiene i campi \textit{cliente\_codice e libro\_isbn}
	\item \textbf{indirizzo:} tabella con campi \textit{codice, via, città, cap, num\_civico, utente} a cui corrisponde l’indirizzo
	\item \textbf{recensioni:} tabella che contiene le recensioni dell'utente. Campi \textit{id\_utente, libro\_isbn, data\_inserimento, valutazione e commento}
\end{itemize}

\begin{figure}[h]
	\centering
	\includegraphics[scale=0.4]{images/er_db}
	\caption{Schema ER del \textit{database}}
\end{figure}

\subsection{Javascript}
Per il comportamento di determinate pagine del sito, sono stati utilizzati file JavaScript che contengono varie funzioni necessarie per il controllo dell’\textit{input nei form}. Attraverso l’utilizzo di espressioni regolari sono stati effettuati i necessari controlli di validità su tutti gli \textit{input}.

Il controllo è stato effettuato sia nella parte pubblica che nella parte privata ( utente e amministratore). È stato indispensabile l’utilizzo di javascript anche nella configurazione dei bottoni. Ad esempio, è stato creato un apposito script per i bottoni dei \textit{filtri} della ricerca (\textit{Reset, PrezzoMin} e \textit{PrezzoMax}), per il bottone Torna Su e per tutti i bottoni delle altre pagine interattive (\textit{Registrati, Accedi, Acquista, Aggiungi al carrello, Aggiungi alla wishlist, Applica sconto...})

Abbiamo creato uno script anche per la configurazione della \textit{navbar} laterale (\textit{burgerMenu.js}), dove sono state sviluppate le funzioni utili per la sua apertura e la chiusura.

Il pulsante del \textit{burger-menu} e poi anche \textit{X} che serve a chiuderla, sono situati in alto a destra delle pagine.

\subsection{XAMPP}
Il corretto funzionamento della parte dinamica, particolarmente PHP e SQL, è stato verificato tramite \textit{XAMPP}. Questo ha reso possibile testare il sito con dispositivi mobile, aprendo la porta 80 dalla propria macchina e rendendo il tutto accessibile nella rete locale.

Così facendo è stato possibile testare meglio il \textit{layout} e i bottoni in modo da evitare il \textit{fat finger}.

\subsection{Chrome Inspect}
La modalità Ispeziona di \textit{Chrome} è stata usata maggiormente per testare \textit{CSS} e \textit{HTML} rendendo il processo di \textit{debugging} più facile. Inoltre, grazie alla possibilità di cambiare gli attributi del CSS senza modificarli nei file originali è stato più semplice decidere quale approccio seguire in alcuni casi di indecisione.



\end{document}