\section{Introduzione}\label{sec:intro}

\subsection{Abstract}
L’idea alla base di \textit{Second Read}, sito sviluppato come progetto del corso Tecnologie Web per l’anno accademico 2021-2022, è stata quella di creare un’ampia piattaforma per la vendita di libri di vario genere.

Gli appassionati possono trovare ed acquistare qui tutti i libri di loro interesse, oppure semplicemente esplorare le diverse opzioni offerte.

Il nome \textit{“SecondRead”} è stato scelto per dare l’idea di fondo di questo progetto: uno spazio web dove ognuno possa trovare il libro che desidera leggere (\textit{Read}) e che tornerà di sicuro a visitare per una seconda volta (\textit{Second}).

La pagina è stata pensata e progettata in modo interattivo con lo scopo di soddisfare qualsiasi bisogno dell'utente.

Al suo interno, i libri sono stati classificati in diverse categorie in base al loro contenuto. Un utente o anche un visitatore casuale non registrato può cercare un libro nella pagina Ricerca tramite il suo titolo, nome dell'autore oppure utilizzando direttamente il suo ISBN.

A tutti i visitatori viene offerto la possibilità di registrarsi gratuitamente e creare il proprio account personale, con il quale possono poi interagire col sito e sfruttare le diverse funzionalità offerte.

In modo da rendere \textit{SecondRead} un sito web vero e proprio, durante la fase di sviluppo si è voluto prestare grande attenzione alle componenti principali di una pagina come l'utilizzabilità, il rispetto degli standard \textit{W3C} e la separazione adeguata tra struttura, presentazione e comportamento tenendo sempre conto delle regole di accessibilità.