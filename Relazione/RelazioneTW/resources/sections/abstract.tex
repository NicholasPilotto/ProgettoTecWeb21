\section{Abstract}\label{sec:abstract}

\subsection{Il progetto}
L’idea dietro \textit{Second Read}, svolto come progetto del corso \textit{Tecnologie Web}, nell’anno accademico 2021-2022, è stata quella di creare un’ampia piattaforma per la vendita di libri di vario genere.

Gli appassionati possono trovare ed acquistare qui tutti i libri di loro interesse, oppure semplicemente esplorare le diverse opzioni offerte.

Il nome \textit{“SecondRead”} è stato scelto per dare l’idea di fondo di questo progetto: un spazio \textit{web} dove ognuno possa trovare il libro che tanto vuole leggere (\textit{Read}) e dove tornerà di sicuro a visitare per una seconda volta (\textit{Second}).

La pagina è stata pensata e progettata in modo interattivo, soddisfando qualsiasi bisogno dell'utente. All'interno, i libri sono stati classificati in diverse categorie in base al loro contenuto. Un utente o anche un visitatore casuale non registrato può cercare un libro nella pagina Ricerca tramite il suo titolo, nome dell'autore oppure utilizzando direttamente il suo codice \textit{ISBN}.

A tutti i visitatori viene offerto la possibilità di registrarsi gratuitamente e creare il proprio \textit{account} personale, con il quale possono poi interagire col sito e sfruttare le diverse funzionalità offerte.

In modo da rendere \textit{SecondRead} un sito \textit{web} vero e proprio, durante la fase di sviluppo si è voluto prestare grande attenzione alle componenti principali di una pagina come l'utilizzabilità, il rispetto degli standard \textit{W3C} e la separazione adeguata tra struttura, presentazione e comportamento tenendo sempre conto delle regole di accessibilità.