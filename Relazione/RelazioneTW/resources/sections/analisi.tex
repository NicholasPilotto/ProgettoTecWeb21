\section{Analisi}

\subsection{Analisi delle caratteristiche utenti}
\textit{SecondRead} rappresenta una piattaforma intermediaria per la vendita dei libri tra individui e le grandi case editrici. L’utenza primaria del sito sarà quindi composta da lettori, siano essi occasionali o appassionati.

Questo non esclude tuttavia il fatto che anche un soggetto legato all’ambito accademico come un professore oppure uno studente possa utilizzare la pagina per trovare libri riguardanti il proprio corso di studio.

Cercando di includere ogni ambito di interesse per i lettori, sono state create inizialmente circa 12 diverse categorie di libri. Si cerca sempre di avere una quantità sufficiente di copie disponibili per ogni libro in modo da poter offrire agli utenti del sito un servizio impeccabile.

Il sito offre all’utente la possibilità di svolgere una ricerca veloce e precisa, effettuata di solito dalle persone che sanno già cosa vogliono, ma anche la possibilità di svolgere ricerche ad ampio spettro che riguardano argomenti correlati a quello inizialmente cercato.

Un altro obiettivo è stato quello di includere nel \textit{target group} tutte le diverse fasce d’età, motivo per il quale si è optato per un linguaggio semplice, informale e intuitivo che possa essere comprensibile a tutti.

La stessa filosofia è stata applicata anche per quanto riguarda struttura e layout del sito: esso è veloce, intuitivo e responsive in modo da adattarsi a vari tipi di dispositivi. Tutto questo rispettando le principali convenzioni del web.

\subsection{Possibile ricerche sui motori di ricerca}
Elencate di seguito, in ordine di rilevanza (da particolare al generale), le possibili ricerche che dovrebbero presentare tra i risultati il sito web.

La ricerca a cui deve sicuramente rispondere il sito è quella contenente il nome stesso, ossia \textit{SecondRead}. Queste ricerche saranno maggiormente effettuate da utenti del sito, da persone che lo hanno visitato precedentemente oppure da persone a cui è stato raccomandato o riferito, ad esempio da un amico.

Essendo un sito di libri online, è fondamentale che risponda in una ricerca a tutte le \textit{query} che contengono un \textit{ISBN, un titolo di libro, oppure un nome di autore o casa editrice} dei libri che sono presenti nella pagina. Questo tipo di ricerca viene effettuata dalle persone che stanno cercando un particolare libro come per esempio uno studente, un genitore o anche un professore universitario.

È importante che il sito risponda anche alle ricerche ad ampio spettro. Ad esempio, se un appassionato di scienza cerca “Scienza e Fantascienza” oppure un appassionato dei manga cerca \textit{“Fumetti e Manga”}, essendo queste categorie di libri parte del sito, esso dovrebbe apparire tra i risultati della ricerca.

Uno degli obiettivi principali del team è quello di espandere la clientela ed attirare nuovi utenti. In modo da realizzarlo, è necessario che il sito risponda anche alle ricerche generali che possono contenere parole chiave come: \textit{libri, libro online, vendita libri, dove comprare libri online,..} . Queste sono le ricerche che vengono effettuate di solito da persone che non hanno esperienza e preferenze specifiche ma che vogliono iniziare a leggere oppure hanno recentemente sviluppato la lettura come \textit{hobby}.

Un altro caso che aiuterebbe ad aumentare la visibilità, è quello in cui la ricerca contiene parole chiave come: libri bestseller, libri scontati online, nuove uscite, prezzo sotto 5 euro...

Una persona che vorrebbe fare un regalo senza spendere troppo sarebbe il caso di utilizzo di questo tipo di ricerca.

In generale, essendo il sito progettato in modo da poter essere utilizzato da tutte le fasce d'età, si è prestato attenzione alle ricerche effettuabili da utenti appassionati ed esperti, ma anche a quelle effettuate da principianti.

\subsection{Conclusioni}
Partendo dall’analisi e dalle considerazioni fatte precedentemente riguardanti il sito e gli utenti, otteniamo le seguenti conclusioni:
\begin{itemize}
	\item \textit{Utenza finale}: gli appassionati di libri e persone legate all'ambito 	accademico
	\item \textit{Target group}: tutte le varie fasce d’età
	\item Ricerca: è possibile effettuare vari tipi di ricerca. Nel caso di ricerca
			specifica, si può utilizzare direttamente la barra di ricerca, mentre per
			quella ad ampio spettro si possono trovare informazioni nei raggruppamenti di
			generi o fasce di prezzo
	\item \textit{Interazione}: all’interno del sito è possibile interagire in diversi
			modi. Un esempio di
			inserimento di dati si trova nella creazione di un account oppure 
			nell'aggiunta di un indirizzo. Un esempio di cancellazione di dati lo 
			proviamo nella cancellazione di recensioni memorizzate precedentemente.
			Si ha anche la possibilità di modificare i dati inseriti come la password,
			l’username oppure l’email.
	\item \textit{Community}: le recensioni sono un elemento pensato ed aggiunto al sito 	in modo da poter creare una forma solida di comunicazione ed interazione tra 	gli utenti. Leggendo le diverse opinioni presenti sulla pagina di un
			determinato libro, l’utente crea un'idea generale più chiara del libro e
			decide se acquistarlo.
\end{itemize}

\subsection{Utenti}
Abbiamo 3 diverse tipologie di utenti:
\begin{enumerate}
	\item \textit{Utente generico(generale)}: Un utente viene considerato generico quando:
		\begin{itemize}
			\item Non è stato registrato, ossia non ha creato il suo account sul sito.
			\item Possiede un account ma non ha ancora effettuato l’accesso.
		\end{itemize}
		Questa tipologia di utente ha accesso a buona parte del sito ma subisce restrizioni su alcuni servizi offerti. Un utente generico non può, ad esempio, fare ordini oppure lasciare recensioni. Tuttavia è comunque in grado di:
			\begin{enumerate}[label*=\arabic*.]
				\item Visualizzare la Homepage del sito
				\item Visualizzare la pagina “Ricerca”
				\item Visualizzare la pagina “Generi”
				\item Visualizzare la pagina “Bestseller”
				\item Visualizzare la pagina “Offerte”
				\item Visualizzare l’elenco dei libri listati in ogni categoria
				\item Cercare un libro nella pagina di ricerca
				\item Visualizzare tutti i dettagli di un particolare libro in vendita 
				\item Registrarsi
				\item Effettuare l’accesso
			\end{enumerate}
	\item \textit{Utente loggato}: Rappresenta l’utente che ha effettuato l’accesso al suo account personale. Eredita tutti i casi d’uso dell’utente generico tranne la parte di registrazione e dell’accesso. Dispone di altre funzionalità aggiuntive come:
			\begin{enumerate}[label*=\arabic*.]
				\item Visitare l'Area Privata 
				\item Visualizzare i suoi ordini
				\item Gestire gli indirizzi
				\item Gestire le recensioni
				\item Visualizzare e modificare le informazioni di login 
				\item Gestire ed aggiornare la propria \textit{wishlist}
				\item Effettuare un ordine
				\item Lasciare una recensione
				\item Contattare il numero di assistenza in caso di problemi
			\end{enumerate}
	\item \textit{Amministratore(admin)}: Rappresenta l’utente che gestisce il sito ed è identificato dall’indirizzo email: \underline{admin@gmail.com}. Possiede la maggior parte delle funzionalità degli utenti sopra elencati ma con alcune eccezioni. Non può ad esempio:
		\begin{itemize}
			\item Lasciare una recensione
			\item Aggiungere elementi nella \textit{wishlist} oppure nel carrello
			\item Effettuare un ordine
		\end{itemize}
Dopo aver effettuato l’accesso al suo account, attraverso l’Area Privata sulla navbar, può eseguire le seguenti funzionalità:
		\begin{enumerate}[label*=\arabic*.]
			\item Aggiungere nuovi libri
			\item Modificare dettagli di un libro già presente nella pagina 
			\item Visualizzare i dati analitici del sito
		\end{enumerate}
\end{enumerate}