\section{Analisi}
\textit{SecondRead} rappresenta una piattaforma intermediaria per la vendita dei libri tra individui e le grandi case editrici. Saranno quindi i lettori appassionati dei libri e che desiderano approfondire le loro conoscenze, l’utenza finale primaria del sito.

Questo non esclude il fatto che anche un soggetto legato all’ambito accademico come un professore oppure uno studente, possa benissimo utilizzare la pagina per trovare libri riguardanti il proprio corso di studio.

Cercando di includere ogni ambito di interesse per i lettori, sono state create inizialmente circa 12 diverse categorie di libri. Si cerca sempre di avere una quantità sufficiente di copie disponibili per ogni libro in modo da poter offrire agli utenti del sito un servizio impeccabile.

Il sito offre all’utente la possibilità di svolgere una ricerca veloce e precisa, effettuata di solito dalle persone che sanno già cosa vogliono, ma anche la possibilità di effettuare ricerche di ampio spettro che possono riguardare argomenti correlati a quello inizialmente cercato.

Un’altro obiettivo è stato quello di includere nel \textit{target group} tutte le diverse fasce di età, motivo per il quale si è optato per un linguaggio semplice, informale e intuitivo da poter essere comprensibile a tutti.

La stessa cosa si è fatta anche per la struttura e il \textit{layout} del sito: essoè  veloce, intuitivo e \textit{responsive} in modo da adattarsi a vari tipi di dispositivi. Tutto questo in rispetto delle principali convenzioni del \textit{web}.

\subsection{Conclusioni}
Partendo dall’analisi e dalle considerazioni fatte precedentemente riguardanti il sito e gli utenti, otteniamo le seguenti conclusioni:
\begin{itemize}
	\item \textit{Utenza finale}: gli appassionati di libri ma anche persone legate all'ambito accademico
	\item \textit{Target group}: tutte le varie fasce d’età
	\item Ricerca: è possibile effettuare vari tipi di ricerca. Nel caso di ricerca specifica, si potrebbe
utilizzare direttamente la barra di ricerca, mentre per quella ad ampio spettro si possono
trovare informazioni nei raggruppamenti in diversi generi.
	\item \textit{Interazione}: all’interno del sito è possibile interagire in diversi modi. Un esempio di inserimento di dati è quello della creazione di un \textit{account} oppure dell'aggiunta di un indirizzo. Un esempio di cancellazione di dati lo troviamo nel caso in cui si volesse cancellare una recensione memorizzata precedentemente. Si ha anche la possibilità di modificare i dati inseriti come la \textit{password}, l’\textit{username} oppure la \textit{mail}.
	\item \textit{Community}: le recensioni sono un elemento pensato ed aggiunto al sito in modo da poter creare una forma solida di comunicazione ed interazione tra gli utenti. Leggendo le diverse opinioni presenti sulla pagina di un determinato libro, si può creare un'idea generale più chiara della trama e magari si è più sicuri se si vuole acquistare o meno.
\end{itemize}

\subsection{Utenza}
Abbiamo 3 diverse tipologie di utenti:
\begin{enumerate}
	\item Utente generico(generale): Un utente viene considerato generico quando:
	\begin{itemize}
		\item Non è stato registrato, ossia non ha creato il suo account sul sito.
		\item Possiede un account ma non ha ancora effettuato l’accesso.
	\end{itemize}
Questa tipologia di utente ha accesso a quasi tutte le sezioni del sito però in modo ristretto. Un utente generico non può, ad esempio, fare ordini oppure lasciare recensioni. Dall’altra parte può benissimo:
	\begin{enumerate}[label*=\arabic*.]
		\item Visualizzare la \textit{Homepage} del sito
	 	\item Visualizzare la pagina “Ricerca”
		\item Visualizzare la pagina “Generi”
		\item Visualizzare la pagina “\textit{Bestseller}”
		\item Visualizzare la pagina “Offerte”
		\item Visualizzare l’elenco dei libri listati in ogni categoria 2.7 
		\item Cercare un libro nella pagina di ricerca
		\item Visualizzare tutti i dettagli di un particolare libro in vendita 2.9
		\item Registrarsi
		\item Effettuare l’accesso
	\end{enumerate}
	\item \textit{Utente loggato}: Rappresenta l’utente che ha effettuato l’accesso al suo account personale. Eredita tutti i casi d’uso dell’utente generico tranne la parte di registrazione e dell’accesso. Dispone di altre funzionalità in più come:
	\begin{enumerate}[label*=\arabic*.]
		\item Visitare l'Area Privata
		\item Visualizzare i suoi ordini
		\item Gestire gli indirizzi
		\item Gestire le recensioni
		\item Visualizzare e modificare le informazioni di \textit{login}
		\item Gestire ed aggiornare la propria \textit{wishlist}
		\item Effettuare un ordine
		\item Lasciare una recensione
		\item Contattare il numero di assistenza in caso di problemi
	\end{enumerate}
	\item \textit{ Amministratore(admin)}: Rappresenta l’utente che gestisce il sito ed è identificato dall’indirizzo email: \textit{admin@gmail.com}. Possiede la maggior parte delle funzionalità degli utenti sopra elencati ma con alcune eccezioni. Non può ad esempio:
		\begin{itemize}
			\item Lasciare una recensione
			\item Aggiungere elementi nella wishlist oppure nel carrello
			\item Effettuare un ordine
		\end{itemize}
Dopo aver effettuato l’accesso al suo \textit{account}, attraverso l’Area Privata sulla \textit{navbar}, può eseguire le seguenti funzionalità:
	\begin{enumerate}[label*=\arabic*.]
		\item Aggiungere nuovi libri
		\item Modificare dettagli di un libro già presente nella pagina 
		\item Visualizzare i dati analitici del sito
	\end{enumerate}
\end{enumerate}