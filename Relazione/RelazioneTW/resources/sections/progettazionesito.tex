\section{Progettazione del sito}\label{sec:progettazionesito}

\subsection{Obiettivi}
La fase di progettazione e sviluppo del sito \textit{SecondRead} è stata fatta tenendo sempre in conto alcuni obiettivi e principi essenziali di un sito ben organizzato e accessibile.
\begin{itemize}
	\item \textit{Separazione contenuto, presentazione e struttura}: Uno degli obiettivi più importanti nella progettazione. 
	
		Il contenuto è stato realizzato tramite documenti \textit{HTML5} e \textit{PHP}. La parte della presentazione grafica è realizzata attraverso i file esterni \textit{CSS} che vengono poi richiamati nelle pagine \textit{HTML}. La realizzazione del comportamento avviene tramite gli \textit{script} \textit{Javascript}. Tutto il codice delle pagine è stato scritto secondo le raccomandazioni \textit{W3C} e in ogni fase della progettazione si è verificato che tutto andasse bene tramite la validazione.
	\item \textit{Accessibilità}: L’accessibilità è uno dei punti fondamentali da tenere in mente durante la progettazione. Si deve realizzare un sito che possa essere utilizzabile da tutte le categorie di utenti e da tutti i tipi di dispositivi e motori di ricerca. In modo da raggiungere questo obiettivo, abbiamo utilizzato varie tecniche.
		\begin{enumerate}
			\item Sono state fornite equivalenti testuali leggibili dallo \textit{screen reader} per tutti i tipi di media come supporto per gli utenti con disabilità visiva.
			\item Grazie all’utilizzo delle misure relative o in percentuale nei fogli \textit{CSS} otteniamo un sito con un design fluido dove ogni pagina è responsive, ossia adattabile ai schermi di diverse dimensioni utilizzati dagli utenti.
			\item Sono stati usati gli attributi \textit{lang} per le parole in lingue diverse dall’italiano in modo da essere lette correttamente dallo \textit{screen reader}.
			\item La scelta dei colori del sito è stata fatta in modo da garantire un contrasto adeguato tra i diversi elementi strutturali e facilitare la lettura del contenuto a chi soffre di disturbi visivi.
			\item Con lo scopo di rendere il sito facile da usare, e cercando di aiutare l’utente alla creazione di una mappa mentale, si sono rispettate alcune convenzioni importanti del web come il colore dei \textit{link}. I \textit{link} non visitati sono sottolineati e in color blu mentre quelli visitati diventano viola.
			\item Si è aggiunto il testo \textit{“Ti trovi in :”} che precede in \textit{breadcrumb} per aiutare gli utenti svantaggiati.
			\item Abbiamo aggiunto un pulsante in basso chiamato \textit{“Torna su”} che se cliccato, effettua uno \textit{scroll} verso l’alto fino all’inizio della pagina, migliorando così l’esperienza durante lo \textit{scroll} delle pagine lunghe.
			\item Utilizzo del \textit{tag abbr} per definire le abbreviazioni.
			\item Si è utilizzato un linguaggio semplice, informale e intuitivo da poter essere
				comprensibile a tutti e in modo da aiutare l’utente a navigare all’interno della pagina corrente oppure tra le diverse pagine.
		\end{enumerate}
	\item \textit{Layout}: Il sito è stato progettato con un layout a tre pannelli:
		\textit{Header}, \textit{Navbar}, Contenuto.
	\item \textit{Resize Mobile}
	\item \textit{Schema organizzativo}
\end{itemize}