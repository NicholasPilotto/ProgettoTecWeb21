\section*{Introduktion: om begrepp som används i mallen}

I mallen använder vi ett antal begrepp som det är viktigt att ha klart för sig vad de avser och hur de relaterar till varandra.
Vi illustrerar detta med exempel.
Du kan ha fått ditt examensarbete som ett uppdrag från t ex ett företag.
I så fall har du ofta fått ett problem som företaget upplever och som du ska försöka hitta lösning på. 
Problemet utgör i detta fall bakgrunden till syftet med arbetet och den frågeställning som du arbetar fram.

Exempel: Företaget X har ett system som de vill kunna använda i en realtidstillämpning, men prestanda i systemet är okänt. 
Problemet är då: Prestanda i systemet är okänt.
Lösningen på problemet är att mäta prestanda.
Syftet med ditt arbete blir att kartlägga systemets prestanda så att du får ett mått på detta. 
Frågeställningen kan formuleras som: Vad är systemets prestanda? Motivationen för arbetet är att det är viktigt att känna prestanda när systemet ska användas för realtidstillämpningar.
När du har syfte och frågeställning klar formulerar du de mål som du ska uppfylla med arbetet, i det här fallet kan målen t ex vara att mäta ett antal olika aspekter av prestanda. 
Tillsammans kommer dessa mål då att uppfylla syftet. 
Men ditt examensarbete behöver inte vara formulerat som ett specifikt problem som ska lösas.
Andra exempel på arbeten som kan förekomma som examensarbeten kan vara:
\begin{itemize}
\item[--]	``Case study'' eller studie av något fenomen
\item[--]	Litteraturstudie
\item[--]	Undersöka något, t ex hur användare interagerar med en mjukvara eller hur en design kan anpassas till en viss grupp användare
\item[--]	Analysera t ex jämföra prestanda hos olika programvaror
\item[--]	Utvärdera hänger ofta ihop med att analysera något, din uppgift kan vara att lämna en rekommendation om vilket verktyg som bäst lämpar sig för en viss uppgift
\item[--]	Utforska ny teknik eller nya angreppssätt. I detta kan ingå att utveckla en artefakt, t ex en mjukvara eller ett system. 
\item[--]	Utreda en frågeställning, t ex genom att göra en förstudie
\item[--]	Utveckla och utvärdera en algoritm, t ex för ett beräkningsproblem
\end{itemize}
Naturligtvis kan ditt examensarbete också innehålla flera av ovanstående komponenter. 
Gemensamt för alla examensarbeten är att de ska vara grundligt vetenskapligt förankrade, ett examensarbete får t ex inte vara enbart en implementation.

I många av exemplen ovan finns det inte något tydligt specificerat problem som ska lösas. 
Det kan istället röra sig om en fråga som du söker svar på, som i exemplet med utvärdering.
Men alla examensarbeten ska ha syfte, frågeställning och motivation. 
Frågeställningen ska vara utformad så att den går att besvara på något sätt genom det arbete du gör.
Men svaret kan vara abstrakt, det kan t ex vara att bidra till kunskap om frågeställningen.
I exemplet där uppgiften är att utforska en teknik, så skulle frågeställningen kunna vara ”Vilka problem finns med att utveckla XX”?
Det är också vanligt att arbetet innebär att du ska analysera och/eller utvärdera artefakten du utvecklat.
Frågeställning och syfte ska matcha varandra på så sätt att när syftet är uppnått så besvaras frågeställningen.
I texten kommer vi att använda begreppet uppgift för det du ska göra, vare sig det är ett problem som ska lösas eller något annat.
Texten i mallen är i denna version på svenska, men för varje rubrik finns motsvarande engelska ord inom parentes.

Denna sida ska inte ingå i slutrapporten
\newpage